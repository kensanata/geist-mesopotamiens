\documentclass{tufte-handout}

% \geometry{showframe}% for debugging purposes -- displays the margins
\hypersetup{colorlinks} % colored links

\usepackage{wasysym} % for \Square
\usepackage[utf8]{inputenc}
\usepackage[ngerman]{babel}

\renewcommand{\labelitemi}{$-$} % use dash for items

% Set up the images/graphics package
\usepackage{graphicx}
\setkeys{Gin}{width=\linewidth,totalheight=\textheight,keepaspectratio}
\graphicspath{{graphics/}}

\title{Der Geist Mesopotamiens}
\author{Alex Schroeder}

% The following package makes prettier tables. We're all about the
% bling!
\usepackage{booktabs}

% The units package provides nice, non-stacked fractions and better
% spacing for units.
% \usepackage{units}

% The fancyvrb package lets us customize the formatting of verbatim
% environments. We use a slightly smaller font.
% \usepackage{fancyvrb}
% \fvset{fontsize=\normalsize}

% Small sections of multiple columns
\usepackage{multicol}

\begin{document}

\maketitle% this prints the handout title, author, and date

\begin{marginfigure}
  \includegraphics{pazuzu_by_tillinghast23-d37j93m.jpg}
\end{marginfigure}

\begin{abstract}
\noindent Dies sind die Regeln für unsere Mesopotamien Runde -- die
fantastische Welt der Stadtstaaten in Sparta, Athen, Ur, Babylon,
Assur, die Welt des Gilgamesch, des trojanischen Krieges, der Odyssee,
der Metamorphosen -- auf so wenig Platz wie möglich. Der
griechisch-mesopotamische Hintergrund bestimmt vor allem die Auswahl
der Fertigkeiten.
\end{abstract}


\section{Aspekte (Aspects)}

\newthought{Jeder Charakter} zwischen zwei und fünf Aspekte.
\marginnote{Herakles ist ein \textit{Sohn des Zeus},
  \textit{jähzornig} und hat \textit{übernatürliche Kraft}. Zwei
  Aspekte bleiben unbestimmt.} Diese können zusammen mit
\textit{Schicksalspunkten} verwendet werden, um Würfelergebnisse zu
verbessern. Die ersten beiden Aspekte vergibt man wie folgt:

\begin{itemize}\itemsep0pt
\item ein Charakterkonzept: Beruf, Kultur oder Persönlichkeit
\item ein Charakterfehler oder ein Beziehungsproblem
\end{itemize}

\noindent Weitere Aspekte ergeben sich idealerweise aus der
gemeinsamen Vergangenheit mit anderen Charakteren.


\section{Fertigkeiten (Skills)}

\newthought{Bei einem Test} würfelt der Spieler mit vier Fudge
Würfeln\marginnote{Herakles ist vor allem ein Ringer. Damit er nicht
  so leicht verletzt wird, wählen wir Athletik. Weil er den Bogen des
  Apollo trägt, muss er auch gut schiessen können. Da er für seine
  Keule bekannt ist, geben wir ihm die entsprechende Fertigkeit,
  obwohl die sonst kaum einer mehr verwendet.} und addiert die
passende Fertigkeit, um eine vom Spielleiter gesetzte Schwierigkeit zu
erreichen. Das Resultat kann durch den Einsatz von Aspekten und
Schicksalspunkten beeinflusst werden.

\begin{margintable}
  % we know that the margin table is 2in wide
  \begin{tabular}{cp{1.8in}}
    4 & Ringen                                                    \\
    3 & Athletik, Bogen                                           \\
    2 & Schleichen, Kurzschwert, Keule (neu)                      \\
    1 & Entschlossenheit, Muse (Leier), Streitwagen, Überleben    \\
  \end{tabular}
\end{margintable}

Wer keine Fudge Würfel hat, kann auch einen positiven und einen
negativen sechsseitigen Würfel zusammenzählen.

Die benötigte Grundausrüstung für die jeweiligen Fertigkeiten hat man
zu Spielbeginn. Wer einen Gegenstand garantiert haben will, muss
hierfür einen Trick investieren (siehe unten).

Jeder Charakter eine Fertigkeit mit Wert 4, zwei Fertigkeiten mit Wert
3, drei Fertigkeiten mit Wert 2 und vier Fertigkeiten mit Wert 1. Alle
anderen Fertigkeiten werden mit Wert 0 verwendet.

\clearpage

\begin{marginfigure}
\includegraphics{pazuzu_dagger_by_tillinghast23-d392ool.jpg}
\end{marginfigure}

\begin{tabular}{ll}
\multicolumn{2}{c}{Grundwerte}                                          \\
Athletik         & verbessert den Zähler für Gesundheit                 \\
Entschlossenheit & verbessert den Zähler für Fassung                    \\
Mittel           & verbessert den Zähler für Vermögen                   \\
\midrule[0pt]
\multicolumn{2}{c}{Kämpfen}                                             \\
Akrobatik        & Ausweichen, Bewegung auf dem Schlachtfeld            \\
Bogen            & Fernkampf                                            \\
Kurzschwert      & Nahkampf                                             \\
Reiten           & nieder reiten, verfolgen                             \\
Ringen           & auch Boxen, Schlagen                                 \\
Schild           & Verteidigen                                          \\
Schleuder        & Fernkampf                                            \\
Speer            & Nahkampf                                             \\
Speerwurf        & Fernkampf                                            \\
Streitwagen      & nieder fahren, Phalanx durchbrechen                  \\
\midrule[0pt]
\multicolumn{2}{c}{Zur Person}                                          \\
Muse             & Singen, Instrumente spielen, Steinmetz, Töpfern      \\
Handeln          & Verhandeln, Geld verdienen                           \\
Rhetorik         & Leute überzeugen, Diplomatie                         \\
Überleben        & In der Wildnis überleben                             \\
Wissenschaft     & Flaschenzüge, Astronomie, Geometrie, Algebra         \\
Mystik           & Götter, Orakel, Prophezeiungen                       \\
Schleichen       & Leise sein, sich verstecken                          \\
Alchemie         & Gift, Heilmittel, Blei zu Gold, griechisches Feuer   \\
Steuermann       & Kurs finden, Strömungen erkennen                     \\
Schiffsbau       & Löcher ausbessern, neue Schiffe bauen, Mast erneuern \\
\end{tabular}

\bigskip
\noindent Normalerweise ist es kein Problem, wenn Spieler weitere
Fertigkeiten ins Spiel bringen.


\section{Zähler (Stress Tracks)}

\newthought{Jeder Charakter} hat drei Zähler mit je zwei Kästchen. Zu
jedem Zähler gehört eine Fertigkeit, mit der man die Anzahl Kästchen
erhöhen kann.

\marginnote[\baselineskip]{Herakles hat Athletik 4 und
  Entschlossenheit 2, deswegen hat er nicht überall zwei Kästchen:}

\begin{margintable}
  \begin{tabular}{ll}
    Gesundheit & \(\Square \Square \Square \Square\) \\
    Fassung    & \(\Square \Square \Square\)         \\
    Vermögen   & \(\Square \Square\)                 \\
  \end{tabular}
\end{margintable}

\bigskip
\begin{minipage}[t]{0.5\textwidth}
  \begin{tabular}[t]{ll}
    Zähler     & Fertigkeit       \\
    \midrule
    Gesundheit & Athletik         \\
    Fassung    & Entschlossenheit \\
    Vermögen   & Mittel           \\
  \end{tabular}
\end{minipage}
\begin{minipage}[t]{0.5\textwidth}
  \hspace{2em}
  \begin{tabular}[t]{cl}
    Wert   & Kästchen                                    \\
    \midrule
    normal & \(\Square \Square\)                         \\
    1--2   & \(\Square \Square \Square\)                 \\
    3+     & \(\Square \Square \Square \Square\)         \\
%   5      & eine weitere milde Konsequenz für den jeweiligen Zähler \\
  \end{tabular}
\end{minipage}


\section{Tricks (Stunts)}

\newthought{Jeder Charakter} hat drei Tricks.\marginnote{Herakles hat
  die folgenden drei Tricks: Einen magischen \textit{Gegenstand}, den
  Bogen des Apollo mit dem Aspekt ``Giftpfeile der Hydra'', welche
  unheilbare Wunden schlagen; die \textit{Begabung}, Schleichen im
  Kampf gegen Ahnungslose einsetzen zu können; eine
  \textit{Ersatzfähigkeit} für sich selber: so prächtig kann er
  Ringen, dass sein blosser Anblick oft stärker wirkt als jedes Wort
  -- so kann er Ringen anstelle von Rhetorik verwenden.} Ein Trick ist
einer der folgenden Vorteile:

\begin{enumerate}

\item ein \textit{Gegenstand} mit eigenem Aspekt und einer passenden
  Spezialfähigkeit; der Spielleiter kann einem diesen Gegenstand auch
  nicht wegnehmen

\item eine \textit{Begabung}: eine Fertigkeit kann auch dann
  eingesetzt werden, wenn es eigentlich nicht mehr möglich ist, oder
  ohne die Abzüge, welche Andere sonst erleiden würden

\item ein \textit{Ersatztfertigkeit} für den eigenen Charakter: eine
  bestehende Fertigkeit auf Wert 1-2 kann durch die gewählte, höhere
  Fertigkeit ersetzt werden; ohne Schicksalspunkt kann sie mit Wert 3
  eingesetzt werden; mit Schicksalspunkt kann der Wert der höheren
  Fertigkeit übernommen werden

\item eine Ersatzfertigkeit \textit{für andere}: die gewählte
  Fertigkeit kann von Begleitern eingesetzt werden; Höchstwert ist 3

\item eine \textit{Bonusfertigkeit} für andere: die gewählte Fertigkeit
  mit Mindestwert 3 kann dafür eingesetzt werden, um Begleitern einen
  +1 Bonus zu geben

\item ein weiteres \textit{Kästchen} für einen Zähler

\item eine \textit{Zählerfertigkeit}: die einem Zähler zugrunde
  liegende Fertigkeit kann durch eine andere Fertigkeit ersetzt
  werden

\end{enumerate}


\section{Schicksalspunkte (Fate Points)}

\newthought{Jeden Abend} beginnen die Charaktere mit drei
Schicksalspunkten.\marginnote{Herakles ringt gegen den nemëischen
  Löwen. Am Tisch wird entschieden, dies mit einem einfach Test
  gegeneinander zu regeln und diskutiert kurz die Konsequenzen von
  Sieg und Niederlage. Der Löwe würfelt 2 und addiert seine
  Kampfesfertigkeit von 3 zu einem Total von 5. Herakles würfelt eine
  0, addiert sein Ringen von 4 und kommt zu einem Total von 4. Der
  Spieler erzählt, dass Herakles ja \textit{übernatürliche Kraft}
  besitzt, welche er von seinem Vater geerbt hat. Er zahlt einen
  Schicksalspunkt und addiert +2 zu seinem Resultat für ein Total von
  6, und so gewinnt Herakles dank der Macht des Schicksals. Es bleiben
  ihm zwei Schicksalspunkte für den Rest des Abends.} Diese setzt man
nach dem Würfeln ein: Man erzählt, warum ein Aspekt die Situation
beeinflusst und wenn alle einverstanden sind, \textbf{addiert man +2}
auf den Wurf oder \textbf{würfelt noch mal}.

Wer dem Spielleiter einen Vorschlag zum eigenen Nachteil vorschlägt,
erhält einen Schicksalspunkt, wenn der Spielleiter darauf eingeht.

Wer einen Vorschlag des Spielleiters zum eigenen Nachteil annimmt,
erhält einen Schicksalspunkt; anderenfalls muss ein Schicksalspunkt
gezahlt werden. Im Kampf nimmt dies meistens die Form von ``setzen
eine Runde aus und nimm dafür einen Schicksalspunkt.''


\section{Kampfrunden}

\marginnote{Herakles ringt mit dem kretischen Stier. Dieser verwüstet
  die ganze Insel: Hörner 4, Trampeln 3, Gesundheit \(\Square
  \Square\). Es gibt den lokalen Aspekt \textit{Die Sonne brennt vom
    Himmel}. Herakles hat Ringen 4, Athletik 3 und zwei
  Schicksalspunkte. Der dritte Schicksalspunkt wurde im letzten
  Beispiel schon gebraucht.

\textbf{Erste Runde}: Herakles würfelt 0, addiert Ringen 4, gibt 4.
Der Stier würfelt -1, addiert Hörner 4, gibt 3. Herakles entscheidet
sich, die brennende Sonne zu seinem Vorteil zu verwenden und zahlt
einen Schicksalspunkt für +2, gibt 6. Die Differenz ist nun 3
zugunsten von Herakles. Die Differenz führt zu einem zusätzlichen,
temporären Aspekt: Der Stier ist ``verwirrt''. Der Stier nimmt eine
milde (-2) Konsequenz und endet \textit{im Würgegriff}; zudem ist
seine Gesundheit \(\XBox \Square\). Der Stier greift zurück an und
würfelt 1, addiert Hörner 4, gibt 5. Herakles würfelt -2. Der Spieler
meint, dass Herakles der \textit{Sohn des Zeus} ist, und seines Vaters
Segen ihm hilft. Er zahlt den letzten Schicksalspunkt und würfelt -3!
Pech gehabt. Der Spieler beschliesst, den freien Aspekt ``verwirrt''
zu verwenden und würfelt -1, addiert Ringen 4, und verwendet den
zweiten freien Aspekt ``im Würgegriff'' für +2, gibt 5. Gleichstand!
Ein temporärer Aspekt ist für den angreifenden Stier entstanden:
Herakles ist \emph{niedergeworfen}.

\textbf{Zweite Runde}: Herakles greift an und würfelt 0, addiert
Ringen 4, gibt 4. Der Stier würfelt -3, addiert Hörner 4 und verwendet
den Aspekt \emph{niedergeworfen} für +2, gibt 3. Die Differenz ist 1
zugunsten von Herakles. Da die Gesundheit des Stiers schon \(\XBox
\Square\) ist, muss trotzdem das zweite Kästchen genommen werden:
\(\XBox \XBox\). Der Stier greift zurück an und würfelt -1, addiert
Hörner 4, gibt 3. Herkules würfelt 0, addiert Ringen 4, gibt 4. Der
Angriff schlägt fehl.

\textbf{Dritte Runde}: Herakles greift an und würfelt 1, addiert
Ringen 4, gibt 5. Der Stier würfelt 0, addiert Hörner 4, gibt 4. Die
Kästchen sind voll, die milde Konsequenz ist schon genommen: Es muss
eine mittlere (-4) Konsequenz sein: \textit{Die Hörner sind
  abgebrochen}. Der Stier gibt auf und bietet die Gefangenschaft an.
Herakles akzeptiert.}

\newthought{In jeder Runde} kann man sich um eine Zone auf der Karte
bewegen (falls es eine Karte gibt) \textit{und} eine der folgenden
Aktionen durchführen:

\begin{enumerate}
\item \textbf{Überwinden}: ein Hindernis überwinden (z.\,B.~eine
  grössere Distanz mit Akrobatik überwinden)
\item \textbf{Vorteil schaffen}: einen Aspekt einführen, der von der
  eigenen Seite ein Mal frei zu verwenden ist;
\item \textbf{Angreifen} (siehe unten)
\end{enumerate}

\noindent In allen Fällen wird gewürfelt und die passende Fertigkeit
dazu addiert. Auf einer Karte zieht man die Zonendifferent ab. Ohne
Gegner wird gegen 0 gewürfelt; ansonsten würfelt der Gegner mit einer
entsprechenden Fertigkeit dagegen. Der Verteidigungswurf gilt gegen
alle Angriffe dieser Runde.

Gelingt die Aktion mit einer Differenz von mindestens drei, so
entsteht ein neuer, temporärer Aspekt, den man ein Mal frei verwenden
darf. Bei Gleichstand entsteht auch ein neuer, temporärer Aspekt, den
der Handelnde oder der Angreifer ein Mal frei verwenden darf.

Bei einem sozialen Konflikt kann man mit \textit{Überwinden} auch
andere Personen auf der Karte bewegen. 

Ist die Differenz zum Vorteil des Angreifers, ist dieser Wert auf dem
entsprechenden Zähler einzutragen. Ist das entsprechende Kästchen
voll, muss ein höheres Kästchen genommen werden. Gibt es kein freies
Kästchen mehr, müssen Konsequenzen genommen werden, um dies zu
kompensieren. Jede Konsequenz nur ein mal genommen werden.

\begin{itemize}
\item \textbf{mild}, bis zu zwei Kästchen, natürliche Erholung ist
  möglich
\item \textbf{mittel}, bis zu vier Kästchen, für die Erholung ist
  Aufwand nötig
\item \textbf{schwer}, bis zu sechs Kästchen, Zeit für die
  Notaufnahme!
\end{itemize}

Konsequenzen sind Aspekte, welche für die Gegenseite ein mal frei zu
verwenden sind. Kann der Schaden nicht mehr kompensiert werden,
scheidet man aus. Man kann dem Gegner auch jederzeit eine Konzession
anbieten und aufgeben.

Nach einer kurzen Verschnaufpause werden die Kästchen aller Zähler
wieder frei. Nur die Konsequenzen bleiben.

\clearpage


\section{Ende des Spielabends}

\marginnote{Herakles hat den Stier gebändigt. Er gibt sich den Aspekt
  \textit{Tierbändiger} und tauscht Mystik mit Muse aus, weil er sich
  seinem göttlichen Erbe näher fühlt und sowieso nie die Leier rührt.
  Er muss keinen Aspekt streichen, weil er ja noch nicht alle fünf
  gewählt hat.}

\newthought{Am Ende des Abends} darf man

\begin{itemize}
\item zwei nebeneinander liegende Fertigkeiten tauschen
\item ein Aspekt austauschen
\item eine milde Konsequenz streichen
\item eine mittlere Konsequenz streichen, die man an einem \textit{früheren}
  Spielabend erlitten hat
\item eine schwere Konsequenz streichen, die man an einem
  \textit{früheren} Spielabend erlitten hat, falls man für diesen
  Zähler dieses mal kein einziges Kästchen verloren hat
\end{itemize}

\section{Lesematerial}

Die historischen Romane von
\href{http://de.wikipedia.org/wiki/Gisbert_Haefs}{Gisbert Haefs} wie
\textit{Troja}, \emph{Alexander} und \emph{Hannibal} geben einen
wunderbaren Einblick in die Antike.

Online findet man immer wieder historische Karten, so zum Beispiel den
\href{http://archive.org/details/AtlasAntiquus}{Atlas Antiquus:
  Taschen Atlas der Alten Welt} von Dr.~Alb.~van~Kampen.

Ebenfalls online, vor allem auf Englisch, oder günstig im Reclam
Verlag zu haben: die \emph{Historien} des
\href{http://de.wikipedia.org/wiki/Herodot}{Herodot}, die
\emph{Anabasis} des Xenophon, und \emph{Der Peloponnesische Krieg} des
\href{http://de.wikipedia.org/wiki/Thukydides}{Thukydides}. Auch die
\emph{Geschichte Alexanders des Grossen} von
\href{http://de.wikipedia.org/wiki/Quintus_Curtius_Rufus_%28Historiker%29}{Quintus
  Curtius Rufus} und den \emph{Alexanderzug} des
\href{http://de.wikipedia.org/wiki/Arrian}{Arrian} kann ich empfehlen.

\clearpage

\section*{CC BY-SA 4.0 License}

\begin{fullwidth}

Der Text steht unter der Creative Commons „Namensnennung - Weitergabe
unter gleichen Bedingungen 4.0 International“ (CC BY-SA 4.0) Lizenz.
Eine Kopie der Lizenz gibt es hier:

\href{https://creativecommons.org/licenses/by-sa/4.0/}{https://creativecommons.org/licenses/by-sa/4.0/deed.de}.

\medskip\noindent Was folgt ist eine einfache Zusammenfassung der (und
kein Ersatz für) die Lizenz.

\medskip\noindent Sie dürfen:

\begin{enumerate}

\item \textbf{Teilen} – das Material in jedwedem Format oder Medium
  vervielfältigen und weiterverbreiten

\item \textbf{Bearbeiten} – das Material remixen, verändern und darauf
  aufbauen

\end{enumerate}

\noindent Der Lizenzgeber kann diese Rechte nicht widerrufen, solange
Sie sich an die Lizenzbedingungen halten.

\begin{enumerate}

\item \textbf{Namensnennung} – Sie müssen angemessene Urheber- und
  Rechteangaben machen, einen Link zur Lizenz beifügen und angeben, ob
  Änderungen vorgenommen wurden. Diese Angaben dürfen in jeder
  angemessenen Art und Weise gemacht werden, allerdings nicht so, dass
  der Eindruck entsteht, der Lizenzgeber unterstütze gerade Sie oder
  Ihre Nutzung besonders.

\item \textbf{Weitergabe unter gleichen Bedingungen} – Wenn Sie das
  Material remixen, verändern oder anderweitig direkt darauf aufbauen,
  dürfen Sie Ihre Beiträge nur unter derselben Lizenz wie das Original
  verbreiten.

\item \textbf{Keine weiteren Einschränkungen} – Sie dürfen keine
  zusätzlichen Klauseln oder technische Verfahren einsetzen, die
  anderen rechtlich irgendetwas untersagen, was die Lizenz erlaubt.

\end{enumerate}

\subsection{Hinweise}

Sie müssen sich nicht an diese Lizenz halten hinsichtlich solcher
Teile des Materials, die gemeinfrei sind, oder soweit Ihre
Nutzungshandlungen durch Ausnahmen und Schranken des Urheberrechts
gedeckt sind.

Es werden keine Garantien gegeben und auch keine Gewähr geleistet. Die
Lizenz verschafft Ihnen möglicherweise nicht alle Erlaubnisse, die Sie
für die jeweilige Nutzung brauchen. Es können beispielsweise andere
Rechte wie Persönlichkeits- und Datenschutzrechte zu beachten sein,
die Ihre Nutzung des Materials entsprechend beschränken.

\subsection{Authors}

Alex Schroeder, \href{https://alexschroeder.ch/}{https://alexschroeder.ch/}

\subsection{Bilder}

Die Bilder von Erik York werden mit freundlicher Genehmigung
verwendet. Sie wurde \emph{für nicht-kommerzielle Zwecke} im Rahmen
von ``Der Geist Mesopotamiens'' gegeben, unter der Bedingung, dass ein
Link auf die Deviant Art des Autors gesetzt wird.

\href{http://tillinghast23.deviantart.com/}{http://tillinghast23.deviantart.com/}

\end{fullwidth}

\nobibliography{}

\end{document}
